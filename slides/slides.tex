%% LaTeX-Beamer template for KIT design
%% by Erik Burger, Christian Hammer
%% title picture by Klaus Krogmann
%%
%% version 2.1
%%
%% mostly compatible to KIT corporate design v2.0
%% http://intranet.kit.edu/gestaltungsrichtlinien.php
%%
%% Problems, bugs and comments to
%% burger@kit.edu

\documentclass[18pt]{beamer}

\usepackage[utf8]{inputenc}
\usepackage[babel,german=quotes]{csquotes}
\usepackage{graphicx}
\usepackage{caption}
\usepackage{subfig}
\usepackage[right]{eurosym}
\usepackage{listings}

%% SLIDE FORMAT

% use 'beamerthemekit' for standard 4:3 ratio
% for widescreen slides (16:9), use 'beamerthemekitwide'

\usepackage{templates/beamerthemekit}
% \usepackage{templates/beamerthemekitwide}

%% TITLE PICTURE

% if a custom picture is to be used on the title page, copy it into the 'logos'
% directory, in the line below, replace 'mypicture' with the 
% filename (without extension) and uncomment the following line
% (picture proportions: 63 : 20 for standard, 169 : 40 for wide
% *.eps format if you use latex+dvips+ps2pdf, 
% *.jpg/*.png/*.pdf if you use pdflatex)

\titleimage{title}

%% TITLE LOGO

% for a custom logo on the front page, copy your file into the 'logos'
% directory, insert the filename in the line below and uncomment it

\titlelogo{titlelogo}

% (*.eps format if you use latex+dvips+ps2pdf,
% *.jpg/*.png/*.pdf if you use pdflatex)

%% TikZ INTEGRATION

% use these packages for PCM symbols and UML classes
% \usepackage{templates/tikzkit}
% \usepackage{templates/tikzuml}

% the presentation starts here

\title[C++ Workshop]{C++ Workshop}
\subtitle{08. Block, 22.06.2012}
\author{Markus Jung, Oliver Schneider, Robert Schneider}

\institute{}

\begin{document}

% change the following line to "ngerman" for German style date and logos
\selectlanguage{ngerman}

\AtBeginSection[]{%
	\begin{frame}
		\tableofcontents[sectionstyle=show/hide,subsectionstyle=hide/show/hide]
	\end{frame}
	\addtocounter{framenumber}{-1}% If you don't want them to affect the slide number
}

%title page
\begin{frame}
\titlepage
\end{frame}

%table of contents
\begin{frame}{Gliederung}
\tableofcontents
\end{frame}

%%%%%%%%%%%%%%%%%%%%%%%%%
% ADD OWN SECTIONS HERE %
%%%%%%%%%%%%%%%%%%%%%%%%%
%\include{cpp} % includes cpp.tex
\section{const}

\subsection{const pointer}

\begin{frame}[fragile]{const variablen}

	\begin{block}{const für Konstanten}
	\begin{lstlisting}[language=C++]
		const int answer = 42;
		answer = 99;           // wird nicht kompiliert
		int j = answer+5;      // ist erlaubt
	\end{lstlisting}
	\end{block}

\end{frame}


\begin{frame}[fragile]{const pointer}

	\begin{block}{const pointer}
	\begin{small}
	\begin{lstlisting}[language=C++]
		// Pointer zu einem konstanten MyClass Objekt
		const MyClass * blub;
		MyClass const * blub2;
		const MyClass const * blub3;

		// konstanter Pointer zu einem MyClass Objekt
		MyClass * const blub4;

		// konstanter Pointer zu einem konstanten MyClass Objekt
		MyClass const * const blub5; 
		const MyClass * const blub6; 
		const MyClass const * const blub7;
	\end{lstlisting}
	\end{small}
	\end{block}

\end{frame}


\begin{frame}[fragile]{const funktionsargumente}

	\begin{block}{const als Veränderungsschutz}
	\begin{small}
	\begin{lstlisting}[language=C++]
		void drucker(int arg)
		{
		    std::cout << arg << std::endl;
		}
		
		void testfun(int *arg)
		{
		    *arg = 42;
		}
		
		void myfun(const int *arg)
		{
		    *arg = 99;     // wird nicht kompiliert
		    testfun(arg);  // wird nicht kompiliert
		    drucker(*arg); // erlaubt
		}
	\end{lstlisting}
	\end{small}
	\end{block}

\end{frame}


\begin{frame}[fragile]{const Funktions-Parameter}

	\begin{block}{const als Absicherung}
		\begin{lstlisting}[language=C++]
			void druckeRechnung(const Rechnung& rech)
			{
			    std::cout << rech->menge << std::endl;
			    rech->menge++; // wird nicht kompiliert
			}
	
			void setzeRechnung(Rechnung& rech)
			{
			    rech->menge = 100;
			}
		\end{lstlisting}
	\end{block}


\end{frame}


\begin{frame}[fragile]{const member funktionen}

\begin{block}{const als Absicherung}
  \lstinputlisting[basicstyle=\tiny]{test.cpp}
\end{block}

\end{frame}

\begin{frame}[fragile]{const member funktionen}

\begin{block}{const als Veränderungsschutz}
\begin{small}
	\begin{lstlisting}[language=C++]
void drucker(const MyClass& arg)
{
  std::cout << arg.getInt() << std::endl;
  arg.setInt(99); // wird nicht kompiliert
}
void testfun(MyClass& arg)
{
  std::cout << arg.getInt() << std::endl;
  arg.setInt(99);
}
void myfun()
{
  MyClass pObj;
  pObj.setInt(42);
  testfun(pObj);
  drucker(pObj);
}
	\end{lstlisting}
	\end{small}
\end{block}

\end{frame}

%
\subsection{mutable}

\begin{frame}[fragile]{const aushebeln}

\begin{block}{mutable}
\begin{small}
	\begin{lstlisting}[language=C++]
class MyClass
{
private:
  mutable unsigned int m_uCounter;
public:
  unsigned int getCounter() const
  {
    return this->m_uCounter;
  }
  int compute(int value) const
  {
    m_uCounter++;
    return value*42;
  }
};
	\end{lstlisting}
	\end{small}
\end{block}

\end{frame}

\subsection{const cast}

\begin{frame}[fragile]{const aushebeln}

\begin{block}{mutable}
\begin{small}
	\begin{lstlisting}[language=C++]
MyClass test;
const MyClass& const_test = test;
std::cout << test.compute(1) << std::endl;
std::cout << test.compute(-3) << std::endl;
std::cout << const_test.compute(-5) << std::endl;
std::cout << test.getCounter() << std::endl; // gibt 3 aus
	\end{lstlisting}
	\end{small}
\end{block}

\end{frame}

\begin{frame}[fragile]{const aushebeln}

\begin{block}{mutable}
\begin{small}
	\begin{lstlisting}[language=C++]
void bad_function(const int& const_ref)
{
  const_ref++; // kompiliert nicht
  int& ref = const_cast<int&>(const_ref);
  ref++; // kompiliert
}

int original = 5;
bad_function(original);
std::cout << original << std::endl; // gibt 6 aus
	\end{lstlisting}
	\end{small}
\end{block}

\end{frame}

\section{Komplexitätstheorie}

\begin{frame}{Vergleich von Algorithmen}
	\begin{block}{Problem}
		Nach welchen Kriterien vergleicht man Algorithmen?
	\end{block}

	\pause

	\begin{itemize}
		\item Rechenzeit
		\item Speicherbedarf
		\item I/O-Bandbreitenbedarf
		\item (Parallelisierbarkeit)
	\end{itemize}

	\pause
	
	Das sind alles von der ausführenden Maschine abhängige Metriken!
	
	Ziel: Ein davon unabhängiger Bewertungsmaßstab \\
	\footnotesize{(Mit dem ferner Beweise und theoretische Analysen möglich sind)}
\end{frame}

\begin{frame}{Big O: Das algorithmische Komplexitätsmaß}
	\begin{block}{Landau-Symbole}
		\begin{itemize}
			\item Beschreiben das asymptotische Verhalten 
			\item Reduktion auf das Wesentliche
			\item Vernachlässigung konstanter Faktoren und asymptotisch unbedeutender Terme
		\end{itemize}
		
		\pause
		
		\begin{itemize}
			\item $f \in \mathcal{O}(g)$ : $f$ wächst asymptotisch höchstens so schnell wie $g$
			\item $f \in \Theta(g)$ : $f$ wächst asymptotisch genau so schnell wie $g$
			\item $f \in \Omega(g)$ : $f$ wächst asymptotisch mindestens so schnell wie $g$
		\end{itemize}
	\end{block}
\end{frame}

\begin{frame}{Landau-Symbole: Anwendung und Beispiele}
	\begin{block}{Notation}
		\begin{itemize}
			\item Landau-Symbole beschreiben eigentlich Mengen von Funktionen
			\item \enquote{Korrekte} Element-Notation: $f \in \mathcal{O}(g)$
			\item Verbreitete Notation: $f = \mathcal{O}(g)$
		\end{itemize}
	\end{block}
	
	\pause
	
		Meist Angabe der Zeitkomplexität in Abhängigkeit der Eingabegröße, aber auch für andere (kritischen) Ressourcen (Speicher, Bandbreite etc.) einsetzbar. Beispiele (Eingabegröße: $n$):
		\begin{itemize}
			\item Arrayzugriffe: $\Theta(1)$
			\item Binäre Suche: $\mathcal{O}(log(n))$
			\item Iterieren durch eine Liste: $\Theta(n)$
			\item Sortieren (vergleichsbasiert): $\Omega(n*log(n))$
		\end{itemize}
\end{frame}



\section{Praxis}
\begin{frame}[fragile]{Praxis!}
	\begin{itemize}
		\item Aufgabe 1: Titel
		\item Aufgabe 2: Noch ein Titel
	\end{itemize}
	\ \\
	\ \\
	\large{\url{https://github.com/kit-cpp-workshop/workshop-ss12-04}} \\
	\ \\
	Aufgabenbeschreibungen und Hinweise: Siehe \verb|README.md|

\end{frame}


\end{document}
