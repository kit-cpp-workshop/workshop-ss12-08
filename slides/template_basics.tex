\section{template basics}


\subsection{Über templates}

\begin{frame}[fragile]{Was sind templates?}
	templates sind: Vorlagen (Schablonen) für Klassen und Funktionen
	
	\pause
	\vspace{2em}
	
	\lstinputlisting[linerange=begin-max_schablone]{cpp-code/function-templates.cpp}
	
	\pause
	
	\begin{lstlisting}
class MyRingbuffer_int { int* p; /*...*/ };
class MyRingbuffer_short;
class MyRingbuffer_double;
class MyRingbuffer_MyType;
	\end{lstlisting}
\end{frame}


\begin{frame}{Zum Einstellen}
	templates sind
	\begin{itemize}
		\item mächtiger als der Präprozessor
		\item schwieriger als der Rest von C++ zusammen
	\end{itemize}
	
	\pause
	\vspace{1em}
	Beides nicht ganz wahr!
	\begin{itemize}
		\item man kann ganz wenige Dinge nur mit dem Präprozessor tun (und viele Dinge nur mit templates)
		\item die grundlegende Verwendung von templates ist einfach
	\end{itemize}
\end{frame}

\begin{frame}{Für den Hinterkopf}
	Wichtig bei templates ist:
	
	\begin{itemize}
		\item templates erzeugen Klassen und Funktionen
		\item aber keinen echten Quelltext
		\item templates werden unmittelbar vor der Übersetzung verarbeitet
		\item Fokus: statische Typsicherheit und Umgang mit Typen
	\end{itemize}
\end{frame}


\subsection{Essentials zur Verwendung}

\begin{frame}[t]{Funktions-templates: Beispiel}
% später: type deduction


	Ziel: Anhand einer Vorlage all diese Funktionen erzeugen \emph{lassen}.\\
	D.h.: Vorlage + Typ $\implies$ Funktion
	
	\vspace{2em}
	
	\onslide*<+> { \lstinputlisting[linerange=begin-max_schablone, commentstyle=\none]{cpp-code/function-templates.cpp} }
	\lstset{morecomment=[l]template}
	\onslide*<+> { \lstinputlisting[linerange=begin-max_template, commentstyle=\none]{cpp-code/function-templates.cpp} }
	\lstset{deletecomment=[l]template}
	\onslide*<+-> { \lstinputlisting[linerange=begin-max_template, commentstyle=\none]{cpp-code/function-templates.cpp} }
\end{frame}

\begin{frame}[fragile]{Funktions-templates: Syntax}
	\begin{block}{Definition eines Funktions-templates}
		\lstinputlisting[linerange=max_template-max_template_short, firstnumber=0]{cpp-code/function-templates.cpp}
	\end{block}
	
	Ein template liefert dem Compiler eine Konstruktionsvorschrift:\\
	Vorlage + Parameter $\implies$ Funktion
	
	\pause
	\vspace{1em}
	
	\begin{block}{Veranschaulichung: int als Parameter T}
		\lstinputlisting[linerange=no_template-no_template_end, firstnumber=last]{cpp-code/function-templates.cpp}
	\end{block}
\end{frame}

\begin{frame}{Verwendung von Funktions-templates}
	Ein Funktions-template ist eine Vorlage, und keine Funktion! (selbes gilt für class templates)\\
	$\implies$ es bedarf des Konstruktionsschrittes \emph{function template} $\rightarrow$ \emph{function}\\
	Nennt sich: Instanziierung (des Funktions-templates)
	
	\vspace{1em}
	
	Warum nicht gleich für alle möglichen Parameter instanziieren?\\
	\pause
	\alert<+>{code bloat!}
	
	\uncover<+->
	{
		\vspace{1em}
		Also nur Instanziieren, wenn tatsächlich benötigt. Funktioniert wie?
	}
	
	\uncover<+->
	{
		\vspace{1em}
		Geschieht \alert{automagisch bei der Verwendung}! Hier liegt der Unterschied zu macros usw.!
	}
\end{frame}

\begin{frame}{Implizite Instanziierung}
	\begin{block}{Aufrufen einer template-Funktion}
		\lstinputlisting[linerange=call0-call1, firstnumber=0]{cpp-code/function-templates.cpp}
	\end{block}
	
	\pause
	
	\begin{itemize}
		\item Compiler erkennt: hier wird eine Schablone benannt und eine Instanz benötigt. \\
		\item $\implies$ Compiler instanziiert das template, Funktionsaufruf wird mit Instanz verbunden.
		\item Das fertige Programm wird an dieser Stelle die entstandene Funktion aufrufen.
	\end{itemize}
	
	\pause
	
	\lstinputlisting[linerange=call1-, firstnumber=last]{cpp-code/function-templates.cpp}
	Eine andere Funktion wird erzeugt!
	
	Es wird nur eine Instanz je Satz von template-Parametern erzeugt pro \emph{translation unit} und nicht für jeden Aufruf eine.
\end{frame}

\begin{frame}[fragile]{Sichtbarkeit und Instanziierung}
	\alert<+>{Bisher:}
	\begin{itemize}
		\item kann eine Funktion aufrufen, wenn die Deklaration vorhanden ist
		\item die Funktion kann (external linkage) in einer anderen translation unit definiert sein
	\end{itemize}
	
	\onslide<+->
	\vspace{0.5em}
	
	\alert<.>{templates:}
	\begin{itemize}
		\item Instanziierung erfordert vorherige \emph{Definition}!
		\item $\implies$ Funktionsaufrufe von noch nicht instanziierten templates erfordern die Definition (den Körper) der Funktion weiter oben in \emph{derselben} translation unit
	\end{itemize}
	
	\onslide<+->
	\vspace{0.5em}
	\begin{description}[leftmargin=3.5em]
		\item[\emph{common solution}] templates inkl. Definitionen in die Header!\\
		\item[\emph{Problem}] One definition rule {\tiny (darf nicht mehr als eine Definition im gesamten Programm haben) }\\
		\item[\emph{Umgehung}] internal linkage (\verb|static|) oder \verb|inline|
	\end{description}
\end{frame}

\begin{frame}{templates \& headers}
	\begin{columns}[t]
		\column{0.5\textwidth}
			\emph{header.h}
			\lstinputlisting[linerange=header-header_end]{cpp-code/template-header.cpp}
		
		\pause
		\column{0.5\textwidth}
			\emph{foobar.cpp}
			\lstinputlisting[linerange=foobar.cpp-main.cpp, firstnumber=0]{cpp-code/template-header.cpp}
			\pause
			\emph{main.cpp}
			\lstinputlisting[linerange=main.cpp-, firstnumber=0]{cpp-code/template-header.cpp}
	\end{columns}
\end{frame}


\subsection{Mehr zu templates}

\begin{frame}[fragile]{template parameters}
	Die template-declaration enthält wie Funktions-Deklarationen eine Menge von Parametern:
	\begin{lstlisting}
		void foo(int, double, MyType, int);
		
		template < typename T, typename MyTemplateParam,
		           class x >
		void bar();
	\end{lstlisting}
	
	Im Kontext einer template-declaration sind \verb|typename| und \verb|class| semantisch identisch.
	
	\pause
	
	Ein template-Parameter empfängt bei der Instanziierung entweder
	\begin{itemize}
		\item einen Typen, d.h. den Namen einer \verb|class|, \verb|struct|, \verb|union| $\Rightarrow$ template type-parameter
		\item oder einen Wert ähnlich wie bei einem Funktionsaufruf $\Rightarrow$ template non-type-parameter \alert{(später!)}
	\end{itemize}
\end{frame}

\begin{frame}[fragile]{class templates}
	Grundsätzlich: wie function templates
	
	\begin{columns}
		\column{0.5\textwidth}
			\begin{lstlisting}
				template < typename T >
				class MyRingbuffer {
				    T* data_member;
				    T memfun(int);
				    /* work with T */
				};
			\end{lstlisting}
			
		\column{0.5\textwidth}
			\begin{lstlisting}
				
				class MyRingbuffer_double {
				    double* data_member;
				    double memfun(int);
				    /* work with double */
				};
			\end{lstlisting}
	\end{columns}
	
	\pause
	\vspace{1em}
	
	Gleiche Probleme:
	\begin{itemize}
		\item Implizite Instanziierung bei Verwendung (z.B. Deklaration eines »Dinges«)
		\item Definition in den Header
		\item member functions entweder \verb|inline| oder die ganze Klasse in einen anonymous namespace ($\implies$ implizit internal linkage)
	\end{itemize}
\end{frame}

\begin{frame}{intricacies: typename, template, dependent names}
\end{frame}
